\documentclass[a4paper,11pt]{article}

\usepackage[spanish]{babel}
\usepackage{graphicx}
\usepackage[ansinew]{inputenc}
\usepackage[utf8]{inputenx}

\title{		\textbf{Informe TP0}}

\author{	Cotarelo Rodrigo, \textit{Padrón Nro. 98577}                     \\
            \texttt{ cotarelorodrigo@gmail.com }                                              \\[2.5ex]
            Etchegaray Rodrigo, \textit{Padrón Nro. 96856}                     \\
            \texttt{ rorroeche@gmail.com }                                              \\[2.5ex]
			Longo Nicolás, \textit{Padrón Nro. 98271}                    
\\
            \texttt{ longo.gnr@hotmail.com }                                              \\[2.5ex]
            \normalsize{1er. Cuatrimestre de 2019}                                      \\
            \normalsize{66.20 Organización de Computadoras  $-$ Práctica Martes}  \\
            \normalsize{Facultad de Ingeniería, Universidad de Buenos Aires}            \\
       }
\date{23/4/19}


\begin{document}

\maketitle
\thispagestyle{empty}   % quita el número en la primer página
\newpage

\section{Enunciado}

\begin{enumerate}
\item Objetivos \\
Familiarizarse con las herramientas de software que usaremos en los siguien-
tes trabajos, implementando un programa (y su correspondiente documenta-
ción) que resuelva el problema piloto que presentaremos más abajo.

\item Alcance \\
Este trabajo práctico es de elaboración grupal, evaluación individual, y de
carácter obligatorio para todos alumnos del curso.

\item Requisitos \\
El trabajo deberá ser entregado personalmente, en la fecha estipulada, con
una carátula que contenga los datos completos de todos los integrantes.
Además, es necesario que el trabajo práctico incluya (entre otras cosas, ver
sección 7), la presentación de los resultados obtenidos, explicando, cuando co-
rresponda, con fundamentos reales, las razones de cada resultado obtenido.

\item Recursos \\
Usaremos el programa GXemul [1] para simular el entorno de desarrollo que
utilizaremos en este y otros trabajos prácticos, una máquina MIPS corriendo
una versión reciente del sistema operativo NetBSD [2].
En la clase del 12/3 hemos repasado, brevemente, los pasos necesarios para
la instalación y configuración del entorno de desarrollo.

\item Programa \\
Se trata de escribir en lenguaje C dos programas que permitan convertir
archivos de texto desde y hacia plataformas UNIX como las que usamos en los
trabajos prácticos. 


\indent
En particular, buscamos generar dos programas: unix2dos para transformar archivs de texto de UNIX a Windows y dos2unix para hacer la operación inversa.


\indent
Por defecto, en ambos casos, tomaremos la entrada estándar stdin y salida por stdout. Las opciones -i y -o nos permiten además indicar explícitamente los archivos de entrada y salida usando - para indicar los flujos estándar que ya mencionmos.


\indent
A continuación se presenta un ejemplo de codificación:

\$	 (echo Uno; echo Dos; echo Tres) \textbar	 unix2dos \textbar	 od -t c

0000000
U
n
o $\backslash$r $\backslash$n
D
o
s $\backslash$r $\backslash$n
T
r
e
s
$\backslash$r
$\backslash$n

0000020

\newpage
\indent Podemos ver, entonces, que por defecto unix2dos opera usando los streams
estandar de entrada/salida, generando los newlines con la convencion adoptada por Windows. Notar ademas el uso del programa od, que nos permite ver caracteres que de otra forma no podrían ser observados.

\indent A continuación, usamos dos2unix para hacer una conversión en el sentido inverso:

\$	od -t c /tmp/dos.txt

0000000
U
n
o $\backslash$r $\backslash$n
D
o
s $\backslash$r
$\backslash$n T r e s $\backslash$r $\backslash$n

0000020

\$ dos2unix -i /tmp/dos.txt -o - \textbar od -t c

0000000
U
n
o $\backslash$n
D
o
s $\backslash$n
T
r
e
s $\backslash$n

0000015

Aqui /tmp/dos.txt contiene un archivo de texto usando la convención
CR+LF para los caracteres newline, y hemos usado el programa dos2unix para
transformarlo a la convensión LF adoptada en sistemas UNIX.

\item Casos de prueba \\
Se deberá verificar el correcto comportamiento del programa, por lo que los
alumnos deberán proponer casos de prueba que crean convenientes, indicando el
motivo de la elección de cada caso, indicando el método utilizado para verificar
que el programa responde correctamente en cada caso.

\item Portabilidad \\
Como es usual, es necesario que la implementación desarrollada provea un
grado minímo de portabilidad. Para satisfacer esto, el programa deberá funcio-
nar al menos en NetBSD/pmax (usando el simulador GXemul [1]) y la versión
de Linux usada para correr el simulador.

\item Informe \\
El informe deberá incluir:

\begin{itemize}
\item Documentación relevante al diseño, implementación, validación y utiliza-
ción del programa.

\item La documentación necesaria para generar los binarios a partir del código
fuente suministrado.

\item Las corridas de prueba, con los comentarios pertinentes.

\item El código fuente, el cual también deberá entregarse en formato digital compilable (incluyendo archivos de entrada y salida de pruebas).

\item Este enunciado.
\end{itemize}

\item Fechas \\
Fecha de vencimiento: martes 16/4 de 2019.

\end{enumerate}

\newpage

\section{Implementacion}
Se implementaron dos funciones para poder convertir un archivo de DOS a UNIX y viceversa. La implementaci\'on se basó en un ciclo while que recorre el archivo caracter por caracter mientras consulta si está parado sobre un '$\backslash$r' o '$\backslash$n'. En caso de ser alguno de estos caracteres, se efectua la modificación correspondiente en el archivo resultado o salida y sino, simplemente se escribe el mismo caracter en el documento de salida.


\section{Errores que surgieron}
Una de las complicaciones que surgieron durante la implementación fue la aparación de un caracter de más en la conversi\'on del archivo. Esto sucedía porque nuestra implementación verificaba si estábamos en el End Of File (EOF) como condición de corte del while para una vez dentro del mismo efectuar la lectura de un nuevo caracter. Lo que suced\'ia, era que en C al leer el \'ultimo caracter se segu\'ia apuntando a este y no al EOF. Por lo tanto, esto implicaba hacer una lectura adicional para llegar al EOF y esto produc\'ia que se escribiese un caracter vac\'io en el documento de salida. Esto se solucion\'o utilizando para la lectura la funci\'on fgetc que lee caracer por caracter. fgetc lee el final del archivo y devuelve el EOF a trav\'es del c\'odigo -1. Por lo tanto, cambiamos nuestra condici\'on de corte del while a esperar que la funci\'on fgetc devolviese el c\'odigo -1.


\section{Validación}
Las primeras pruebas básicas que se hicieron, fueron convertir archivos de texto de un formato a otro chequeando que el archivo de salida tuviese el mismo aspecto que el archivo de entrada. También se hicieron comparaciones entre los archivos de entrada y salida usando el programa \textit{object dump (od)}, para tener una vista un poco más detallada de los documentos y sus caracteres.

\indent Como esto no era suficiente, y las pruebas que deseábamos hacer debían ser más exhaustivas se decidió generar archivos de salida esperados. Es decir, para cada archivo con saltos de línea del tipo LF se generó uno exactamente igual pero con saltos de línea del tipo CR+LF y viceversa. De ésta manera, pudimos comprobar que los archivos generados por el programa son exactamente iguales a aquellos esperados utilizando la directiva \textit{cmp} de la siguiente manera:

\begin{verbatim}
cmp archivoDePrueba01_salidaReal.txt archivoDePrueba02_salidaEsperada.txt
\end{verbatim}

\indent El resultado es el esperado si al ejecutar esa sentencia en la shell la salida es, literalmente, ninguna. De lo contrario, el programa \textit{cmp} habrá encontrado una diferencia entre los caracteres y eso evidenciaría un problema en la salida de nuestro programa conversor.

%¿¿¿¿¿¿¿¿¿ Por último se generaron algunas pruebas en código que comprobasen el correcto funcionamiento de los programas desarrollados.???? Qué dice????


\section{Utilización}
Para ejecutar los programas se utiliza la consola o shell (convencionalmente), donde se corren ambos ejecutables. Dependiendo de qué tipo de conversión de archivo se desee, se usará alguno de los dos programas. La ejecución de cualquiera de estos puede recibir dos parámetros: el archivo de entrada y el archivo de salida. Si no recibe el primero, entonces se setea por defecto la entrada estándar (\textit{stdin}); si no recibe el segundo, se setea por defecto la salida estándar (\textit{stdout}).

\indent	
Antes del nombre del archivo de entrada se debe especificar el flag '-i' y antes del nombre del archivo de salida, el flag '-o'.

\indent	
Por ejemplo, para convertir un archivo de DOS a UNIX se ejecuta: 

\begin{verbatim}
./dos2unix -i archivoDeEntradaWindows.txt -o archivoDeSalidaUnix.txt
\end{verbatim}

\section{Generaci\'on de binarios}
Para la generaci\'on de los binarios a partir de los archivos con el c\'odigo fuente suministrado se deben seguir los siguientes pasos (usaremos de ejemplo el programa unix2dos, pero aplica para ambos programas):

\begin{itemize}
\item Abir la terminal y posicionarse en el directorio donde se encuentren los archivos con el c\'odigo fuente.
\item \begin{verbatim}
gcc -Wall -O0 -o unix2dos unix2dos.c
\end{verbatim} 
\end{itemize}

Para ejecutar los binarios creados, se deben seguir los pasos de la secci\'on de Utilizaci\'on.

Para generar los archivos assembly de MIPS, se debe mover los archivos fuente a trav\'es del tunel ssh y ejecutar el siguiente comando:
\begin{itemize}
\item \begin{verbatim}
gcc -Wall -O0 -S -mrnames unix2dos.c
\end{verbatim}
\end{itemize}

\section{Pruebas}
Con el fin de probar la robustez de los programas creamos distintos archivos para convertir.

\begin{itemize}
\item Para pasar de linux a DOS hicimos los siguientes archivos:
\begin{verbatim}
od -t c linux_prueba_1.txt 
0000000   A   r   c   h   i   v   o       d   e       l   i   n   u   x
0000020  \n   Q   u   e       t   i   e   n   e  \n   M   u   c   h   o
0000040   s       e   n   t   e   r   s  \n   F   a   l   o   p   a   s
0000060  \n   H  \n   O  \n   L  \n   A  \n  \n   P   R   O   H   I   B
0000100   I   D   O       T   I   R   A   R       E   N   T   E   R   S
0000120       D   E       M   I   E   R   D   A  \n  \n   f   i   n    
0000140   d   e   l      \n   a   r   c   h   i   v   o  \n  \n   a   h
0000160   o   r   a       s   i
0000166
\end{verbatim}

El resultado obtenido fue:
\begin{verbatim}
echo | ./unix2dos -i pruebas/linux_prueba_1.txt | od -t c
0000000   A   r   c   h   i   v   o       d   e       l   i   n   u   x
0000020  \r  \n   Q   u   e       t   i   e   n   e  \r  \n   M   u   c
0000040   h   o   s       e   n   t   e   r   s  \r  \n   F   a   l   o
0000060   p   a   s  \r  \n   H  \r  \n   O  \r  \n   L  \r  \n   A  \r
0000100  \n  \r  \n   P   R   O   H   I   B   I   D   O       T   I   R
0000120   A   R       E   N   T   E   R   S       D   E       M   I   E
0000140   R   D   A  \r  \n  \r  \n   f   i   n       d   e   l      \r
0000160  \n   a   r   c   h   i   v   o  \r  \n  \r  \n   a   h   o   r
0000200   a       s   i
0000204
\end{verbatim}

\begin{verbatim}
od -t c linux_prueba_2.txt 
0000000   H  \n   H  \n   H  \n   H  \n   H  \n   H  \n   H  \n   H  \n
0000020   H  \n  \n  \n  \n  \n   F  \n   F  \n   F  \n   F  \n   F  \n
0000040  \n   F  \n   F  \n   F  \n   F  \n  \n   F  \n   F  \n   F  \n
0000060   F  \n   F  \n   F  \n  \n   J  \n   J  \n   J  \n   J  \n   J
0000100  \n   J  \n   J  \n   J  \n   J  \n   K  \n   K  \n   K  \n   K
0000120  \n  \n   C  \n   F  \n   K  \n  \n  \n  \n  \n   O   P   O   P
0000140   O   P   O   P   O   P  \n   O   P   O   P   O   P   O   P   O
0000160   P  \n  \n   O   P   O   P   O   P   O   P   O   P  \n   O   P
0000200   O   P   O   P   O   P   O   P  \n  \n   J  \n  \n  \n
\end{verbatim}

El resultado obtenido fue:
\begin{verbatim}
echo | ./unix2dos -i pruebas/linux_prueba_2.txt | od -t c
0000000   H  \r  \n   H  \r  \n   H  \r  \n   H  \r  \n   H  \r  \n   H
0000020  \r  \n   H  \r  \n   H  \r  \n   H  \r  \n  \r  \n  \r  \n  \r
0000040  \n  \r  \n   F  \r  \n   F  \r  \n   F  \r  \n   F  \r  \n   F
0000060  \r  \n  \r  \n   F  \r  \n   F  \r  \n   F  \r  \n   F  \r  \n
0000100  \r  \n   F  \r  \n   F  \r  \n   F  \r  \n   F  \r  \n   F  \r
0000120  \n   F  \r  \n  \r  \n   J  \r  \n   J  \r  \n   J  \r  \n   J
0000140  \r  \n   J  \r  \n   J  \r  \n   J  \r  \n   J  \r  \n   J  \r
0000160  \n   K  \r  \n   K  \r  \n   K  \r  \n   K  \r  \n  \r  \n   C
0000200  \r  \n   F  \r  \n   K  \r  \n  \r  \n  \r  \n  \r  \n  \r  \n
0000220   O   P   O   P   O   P   O   P   O   P  \r  \n   O   P   O   P
0000240   O   P   O   P   O   P  \r  \n  \r  \n   O   P   O   P   O   P
0000260   O   P   O   P  \r  \n   O   P   O   P   O   P   O   P   O   P
0000300  \r  \n  \r  \n   J  \r  \n  \r  \n  \r  \n
0000313
\end{verbatim}

\item Para el caso de DOS a linux usamos el siguiente archivo:
\begin{verbatim}
od -t c windows_prueba_1.txt 
0000000   N   i   c   o  \r  \n   E   r   t   g   w   y   g   h   q  \r
0000020  \n   q   e   r   h   y   4   q   3   e   a   u   h   q   4   e
0000040   3   r   u   j   q   4   a      \r  \n   3  \r  \n
0000055
\end{verbatim}

Obteniendo como resultado:
\begin{verbatim}
echo | ./dos2unix -i pruebas/windows_prueba_1.txt | od -t c
0000000   N   i   c   o  \n   E   r   t   g   w   y   g   h   q  \n   q
0000020   e   r   h   y   4   q   3   e   a   u   h   q   4   e   3   r
0000040   u   j   q   4   a      \n   3  \n
0000051
\end{verbatim}

\item Luego probamos con un archivo mas grande:
\begin{verbatim}
od -t c windows_prueba_2.txt 
0000000   E   s   t   a       e   s       o   t   r   a       p   r   u
0000020   e   b   a       d   e       W   i   n   d   o   w   s  \r  \n
0000040  \r  \n  \r  \n   E   s   t   a       e   s       o   t   r   a
0000060       p   r   u   e   b   a       d   e       W   i   n   d   o
0000100   w   s  \r  \n  \r  \n  \r  \n  \r  \n  \r  \n   E   s   t   a
0000120       e   s       o   t   r   a       p   r   u   e   b   a    
0000140   d   e       W   i   n   d   o   w   s  \r  \n  \r  \n  \r  \n
0000160   E   s   t   a       e   s       o   t   r   a       p   r   u
0000200   e   b   a       d   e       W   i   n   d   o   w   s  \r  \n
0000220  \r  \n   E   s   t   a       e   s       o   t   r   a       p
0000240   r   u   e   b   a       d   e       W   i   n   d   o   w   s
0000260  \r  \n  \r  \n   E   s   t   a       e   s       o   t   r   a
0000300       p   r   u   e   b   a       d   e       W   i   n   d   o
0000320   w   s  \r  \n  \r  \n   E   s   t   a       e   s       o   t
0000340   r   a       p   r   u   e   b   a       d   e       W   i   n
0000360   d   o   w   s  \r  \n   E   s   t   a       e   s       o   t
0000400   r   a       p   r   u   e   b   a       d   e       W   i   n
0000420   d   o   w   s  \r  \n   E   s   t   a       e   s       o   t
0000440   r   a       p   r   u   e   b   a       d   e       W   i   n
0000460   d   o   w   s  \r  \n   E   s   t   a       e   s       o   t
0000500   r   a       p   r   u   e   b   a       d   e       W   i   n
0000520   d   o   w   s  \r  \n  \r  \n   E   s   t   a       e   s    
0000540   o   t   r   a       p   r   u   e   b   a       d   e       W
0000560   i   n   d   o   w   s   E   s   t   a       e   s       o   t
0000600   r   a       p   r   u   e   b   a       d   e       W   i   n
0000620   d   o   w   s   E   s   t   a       e   s       o   t   r   a
0000640       p   r   u   e   b   a       d   e       W   i   n   d   o
0000660   w   s   E   s   t   a       e   s       o   t   r   a       p
0000700   r   u   e   b   a       d   e       W   i   n   d   o   w   s
0000720   E   s   t   a       e   s       o   t   r   a       p   r   u
0000740   e   b   a       d   e       W   i   n   d   o   w   s  \r  \n
0000760  \r  \n  \r  \n  \r  \n   E   s   t   a       e   s       o   t
0001000   r   a       p   r   u   e   b   a       d   e       W   i   n
0001020   d   o   w   s  \r  \n   E   s   t   a       e   s       o   t
0001040   r   a       p   r   u   e   b   a       d   e       W   i   n
0001060   d   o   w   s  \r  \n   E   s   t   a       e   s       o   t
0001100   r   a       p   r   u   e   b   a       d   e       W   i   n
0001120   d   o   w   s   E   s   t   a       e   s       o   t   r   a
0001140       p   r   u   e   b   a       d   e       W   i   n   d   o
0001160   w   s  \r  \n   E   s   t   a       e   s       o   t   r   a
0001200       p   r   u   e   b   a       d   e       W   i   n   d   o
0001220   w   s  \r  \n  \r  \n   E   s   t   a       e   s       o   t
0001240   r   a       p   r   u   e   b   a       d   e       W   i   n
0001260   d   o   w   s   E   s   t   a       e   s       o   t   r   a
0001300       p   r   u   e   b   a       d   e       W   i   n   d   o
0001320   w   s   E   s   t   a       e   s       o   t   r   a       p
0001340   r   u   e   b   a       d   e       W   i   n   d   o   w   s
0001360   E   s   t   a       e   s       o   t   r   a       p   r   u
0001400   e   b   a       d   e       W   i   n   d   o   w   s   E   s
0001420   t   a       e   s       o   t   r   a       p   r   u   e   b
0001440   a       d   e       W   i   n   d   o   w   s   E   s   t   a
0001460       e   s       o   t   r   a       p   r   u   e   b   a    
0001500   d   e       W   i   n   d   o   w   s   E   s   t   a       e
0001520   s       o   t   r   a       p   r   u   e   b   a       d   e
0001540       W   i   n   d   o   w   s   E   s   t   a       e   s    
0001560   o   t   r   a       p   r   u   e   b   a       d   e       W
0001600   i   n   d   o   w   s   E   s   t   a       e   s       o   t
0001620   r   a       p   r   u   e   b   a       d   e       W   i   n
0001640   d   o   w   s   E   s   t   a       e   s       o   t   r   a
0001660       p   r   u   e   b   a       d   e       W   i   n   d   o
0001700   w   s   E   s   t   a       e   s       o   t   r   a       p
0001720   r   u   e   b   a       d   e       W   i   n   d   o   w   s
0001740   E   s   t   a       e   s       o   t   r   a       p   r   u
0001760   e   b   a       d   e       W   i   n   d   o   w   s   E   s
0002000   t   a       e   s       o   t   r   a       p   r   u   e   b
0002020   a       d   e       W   i   n   d   o   w   s   E   s   t   a
0002040       e   s       o   t   r   a       p   r   u   e   b   a    
0002060   d   e       W   i   n   d   o   w   s   E   s   t   a       e
0002100   s       o   t   r   a       p   r   u   e   b   a       d   e
0002120       W   i   n   d   o   w   s   E   s   t   a       e   s    
0002140   o   t   r   a       p   r   u   e   b   a       d   e       W
0002160   i   n   d   o   w   s   E   s   t   a       e   s       o   t
0002200   r   a       p   r   u   e   b   a       d   e       W   i   n
0002220   d   o   w   s   E   s   t   a       e   s       o   t   r   a
0002240       p   r   u   e   b   a       d   e       W   i   n   d   o
0002260   w   s   E   s   t   a       e   s       o   t   r   a       p
0002300   r   u   e   b   a       d   e       W   i   n   d   o   w   s
0002320   E   s   t   a       e   s       o   t   r   a       p   r   u
0002340   e   b   a       d   e       W   i   n   d   o   w   s   E   s
0002360   t   a       e   s       o   t   r   a       p   r   u   e   b
0002400   a       d   e       W   i   n   d   o   w   s   E   s   t   a
0002420       e   s       o   t   r   a       p   r   u   e   b   a    
0002440   d   e       W   i   n   d   o   w   s  \r  \n  \r  \n   E   s
0002460   t   a       e   s       o   t   r   a       p   r   u   e   b
0002500   a       d   e       W   i   n   d   o   w   s  \r  \n   E   s
0002520   t   a       e   s       o   t   r   a       p   r   u   e   b
0002540   a       d   e       W   i   n   d   o   w   s  \r  \n  \r  \n
0002560  \r  \n   E   s   t   a       e   s       o   t   r   a       p
0002600   r   u   e   b   a       d   e       W   i   n   d   o   w   s
0002620  \r  \n   E   s   t   a       e   s       o   t   r   a       p
0002640   r   u   e   b   a       d   e       W   i   n   d   o   w   s
0002660  \r  \n  \r  \n  \r  \n  \r  \n   E   s   t   a       e   s    
0002700   o   t   r   a       p   r   u   e   b   a       d   e       W
0002720   i   n   d   o   w   s  \r  \n   E   s   t   a       e   s    
0002740   o   t   r   a       p   r   u   e   b   a       d   e       W
0002760   i   n   d   o   w   s  \r  \n  \r  \n  \r  \n  \r  \n  \r  \n
0003000   E   s   t   a       e   s       o   t   r   a       p   r   u
0003020   e   b   a       d   e       W   i   n   d   o   w   s  \r  \n
0003040   E   s   t   a       e   s       o   t   r   a       p   r   u
0003060   e   b   a       d   e       W   i   n   d   o   w   s  \r  \n
0003100  \r  \n  \r  \n  \r  \n  \r  \n  \r  \n  \r  \n
0003114
\end{verbatim}

Obteniendo como resultado:
\begin{verbatim}
echo | ./dos2unix -i pruebas/windows_prueba_2.txt | od -t c
0000000   E   s   t   a       e   s       o   t   r   a       p   r   u
0000020   e   b   a       d   e       W   i   n   d   o   w   s  \n  \n
0000040  \n   E   s   t   a       e   s       o   t   r   a       p   r
0000060   u   e   b   a       d   e       W   i   n   d   o   w   s  \n
0000100  \n  \n  \n  \n   E   s   t   a       e   s       o   t   r   a
0000120       p   r   u   e   b   a       d   e       W   i   n   d   o
0000140   w   s  \n  \n  \n   E   s   t   a       e   s       o   t   r
0000160   a       p   r   u   e   b   a       d   e       W   i   n   d
0000200   o   w   s  \n  \n   E   s   t   a       e   s       o   t   r
0000220   a       p   r   u   e   b   a       d   e       W   i   n   d
0000240   o   w   s  \n  \n   E   s   t   a       e   s       o   t   r
0000260   a       p   r   u   e   b   a       d   e       W   i   n   d
0000300   o   w   s  \n  \n   E   s   t   a       e   s       o   t   r
0000320   a       p   r   u   e   b   a       d   e       W   i   n   d
0000340   o   w   s  \n   E   s   t   a       e   s       o   t   r   a
0000360       p   r   u   e   b   a       d   e       W   i   n   d   o
0000400   w   s  \n   E   s   t   a       e   s       o   t   r   a    
0000420   p   r   u   e   b   a       d   e       W   i   n   d   o   w
0000440   s  \n   E   s   t   a       e   s       o   t   r   a       p
0000460   r   u   e   b   a       d   e       W   i   n   d   o   w   s
0000500  \n  \n   E   s   t   a       e   s       o   t   r   a       p
0000520   r   u   e   b   a       d   e       W   i   n   d   o   w   s
0000540   E   s   t   a       e   s       o   t   r   a       p   r   u
0000560   e   b   a       d   e       W   i   n   d   o   w   s   E   s
0000600   t   a       e   s       o   t   r   a       p   r   u   e   b
0000620   a       d   e       W   i   n   d   o   w   s   E   s   t   a
0000640       e   s       o   t   r   a       p   r   u   e   b   a    
0000660   d   e       W   i   n   d   o   w   s   E   s   t   a       e
0000700   s       o   t   r   a       p   r   u   e   b   a       d   e
0000720       W   i   n   d   o   w   s  \n  \n  \n  \n   E   s   t   a
0000740       e   s       o   t   r   a       p   r   u   e   b   a    
0000760   d   e       W   i   n   d   o   w   s  \n   E   s   t   a    
0001000   e   s       o   t   r   a       p   r   u   e   b   a       d
0001020   e       W   i   n   d   o   w   s  \n   E   s   t   a       e
0001040   s       o   t   r   a       p   r   u   e   b   a       d   e
0001060       W   i   n   d   o   w   s   E   s   t   a       e   s    
0001100   o   t   r   a       p   r   u   e   b   a       d   e       W
0001120   i   n   d   o   w   s  \n   E   s   t   a       e   s       o
0001140   t   r   a       p   r   u   e   b   a       d   e       W   i
0001160   n   d   o   w   s  \n  \n   E   s   t   a       e   s       o
0001200   t   r   a       p   r   u   e   b   a       d   e       W   i
0001220   n   d   o   w   s   E   s   t   a       e   s       o   t   r
0001240   a       p   r   u   e   b   a       d   e       W   i   n   d
0001260   o   w   s   E   s   t   a       e   s       o   t   r   a    
0001300   p   r   u   e   b   a       d   e       W   i   n   d   o   w
0001320   s   E   s   t   a       e   s       o   t   r   a       p   r
0001340   u   e   b   a       d   e       W   i   n   d   o   w   s   E
0001360   s   t   a       e   s       o   t   r   a       p   r   u   e
0001400   b   a       d   e       W   i   n   d   o   w   s   E   s   t
0001420   a       e   s       o   t   r   a       p   r   u   e   b   a
0001440       d   e       W   i   n   d   o   w   s   E   s   t   a    
0001460   e   s       o   t   r   a       p   r   u   e   b   a       d
0001500   e       W   i   n   d   o   w   s   E   s   t   a       e   s
0001520       o   t   r   a       p   r   u   e   b   a       d   e    
0001540   W   i   n   d   o   w   s   E   s   t   a       e   s       o
0001560   t   r   a       p   r   u   e   b   a       d   e       W   i
0001600   n   d   o   w   s   E   s   t   a       e   s       o   t   r
0001620   a       p   r   u   e   b   a       d   e       W   i   n   d
0001640   o   w   s   E   s   t   a       e   s       o   t   r   a    
0001660   p   r   u   e   b   a       d   e       W   i   n   d   o   w
0001700   s   E   s   t   a       e   s       o   t   r   a       p   r
0001720   u   e   b   a       d   e       W   i   n   d   o   w   s   E
0001740   s   t   a       e   s       o   t   r   a       p   r   u   e
0001760   b   a       d   e       W   i   n   d   o   w   s   E   s   t
0002000   a       e   s       o   t   r   a       p   r   u   e   b   a
0002020       d   e       W   i   n   d   o   w   s   E   s   t   a    
0002040   e   s       o   t   r   a       p   r   u   e   b   a       d
0002060   e       W   i   n   d   o   w   s   E   s   t   a       e   s
0002100       o   t   r   a       p   r   u   e   b   a       d   e    
0002120   W   i   n   d   o   w   s   E   s   t   a       e   s       o
0002140   t   r   a       p   r   u   e   b   a       d   e       W   i
0002160   n   d   o   w   s   E   s   t   a       e   s       o   t   r
0002200   a       p   r   u   e   b   a       d   e       W   i   n   d
0002220   o   w   s   E   s   t   a       e   s       o   t   r   a    
0002240   p   r   u   e   b   a       d   e       W   i   n   d   o   w
0002260   s   E   s   t   a       e   s       o   t   r   a       p   r
0002300   u   e   b   a       d   e       W   i   n   d   o   w   s   E
0002320   s   t   a       e   s       o   t   r   a       p   r   u   e
0002340   b   a       d   e       W   i   n   d   o   w   s   E   s   t
0002360   a       e   s       o   t   r   a       p   r   u   e   b   a
0002400       d   e       W   i   n   d   o   w   s  \n  \n   E   s   t
0002420   a       e   s       o   t   r   a       p   r   u   e   b   a
0002440       d   e       W   i   n   d   o   w   s  \n   E   s   t   a
0002460       e   s       o   t   r   a       p   r   u   e   b   a    
0002500   d   e       W   i   n   d   o   w   s  \n  \n  \n   E   s   t
0002520   a       e   s       o   t   r   a       p   r   u   e   b   a
0002540       d   e       W   i   n   d   o   w   s  \n   E   s   t   a
0002560       e   s       o   t   r   a       p   r   u   e   b   a    
0002600   d   e       W   i   n   d   o   w   s  \n  \n  \n  \n   E   s
0002620   t   a       e   s       o   t   r   a       p   r   u   e   b
0002640   a       d   e       W   i   n   d   o   w   s  \n   E   s   t
0002660   a       e   s       o   t   r   a       p   r   u   e   b   a
0002700       d   e       W   i   n   d   o   w   s  \n  \n  \n  \n  \n
0002720   E   s   t   a       e   s       o   t   r   a       p   r   u
0002740   e   b   a       d   e       W   i   n   d   o   w   s  \n   E
0002760   s   t   a       e   s       o   t   r   a       p   r   u   e
0003000   b   a       d   e       W   i   n   d   o   w   s  \n  \n  \n
0003020  \n  \n  \n  \n
0003024
\end{verbatim}
\end{itemize}

Adem\'as como se explic\'o anteriormente, creamos archivos con las salidas esperadas y luego usamos el programa \textit{cmp} para corroborar que no hab\'ia diferencias entre el archivo esperado y el real. 

\section{C\'odigo fuente}
\subsection{unix2dos}
\begin{verbatim}
#include <stdio.h>
#include <stdlib.h>
#include <string.h>

#define MAX_NAME 100

const char CR = '\r';
const char LF = '\n';

int mostrarMensajeErrorParametrosInvalidos()
{
  fprintf(stderr, "Los parámetros ingresados no son válidos.\n");
  return -1;
}

void linuxToWindows(FILE* input_file, FILE* output_file)
{
	int c;
  	while((c = fgetc(input_file)) != -1)
  	{
  		if ( c == LF )
  		{
    		fputc(CR, output_file);
    		fputc(LF, output_file);
    	}
    	else
    	{
    		fputc(c, output_file);
    	}
  	}
}


int main(int argc, char** argv)
{
  // Inicializo los valores por defecto. No se ingresaron archivos y estos son stdin y stdout
  int seIngresoArchivoDeEntrada = 0;
  int seIngresoArchivoDeSalida = 0;
  char *input_fileName = NULL;
  char *output_fileName = NULL;

/* No puedo recibir más de 5 parámetros. Este es el máximo esperado. Por otro lado, puedo
   recibir 1 parámetro (el nombre del programa), 3 parámetros (se especifica archivo de
   entrada o de salida) y 5 parámetros (se especifican ambos). Además argc siempre es mayor
   o igual a 1          */

  if (argc > 5 || argc == 4 || argc == 2) {
    return mostrarMensajeErrorParametrosInvalidos();
  }

  // Caso de recepción de al menos un archivo por parametro
  if (argc >= 3) {
    if (strcmp(argv[1], "-i") == 0) {
      if (strcmp(argv[2], "-") != 0)
      {
        input_fileName = argv[2];
        seIngresoArchivoDeEntrada = 1;
      }
    }
    else if (strcmp(argv[1], "-o") == 0) {
      if (strcmp(argv[2], "-") != 0)
      {
        output_fileName = argv[2];
        seIngresoArchivoDeSalida = 1;
      }
    }
    else {
      // Parámetro invalido
      return mostrarMensajeErrorParametrosInvalidos();
    }
    // Caso de recepción de dos archivos por parámetro
    if (argc > 4) {
      if (seIngresoArchivoDeEntrada) {
        if (strcmp(argv[3], "-o") == 0) {
          if (strcmp(argv[4], "-") != 0)
          {
            output_fileName = argv[4];
            seIngresoArchivoDeSalida = 1;
          }
        }
        else {
          // Si ya se ingreso el parametro "-i" el único válido que queda es "-o"
          return mostrarMensajeErrorParametrosInvalidos();
        }
      }
      else if (seIngresoArchivoDeSalida) {
        if (strcmp(argv[3], "-i") == 0) {
          if (strcmp(argv[4], "-") != 0)
          {
            input_fileName = argv[4];
            seIngresoArchivoDeEntrada = 1;
          }
        }
        else {
          // Si ya se ingreso el parametro "-o" el único válido que queda es "-i"
          return mostrarMensajeErrorParametrosInvalidos();
        }
      }
    }
  }

  FILE* input_file;
  if (input_fileName != NULL)
  {
    if( (input_file = fopen(input_fileName, "rt")) == NULL)
    {
        fprintf(stderr,"No se pudo abrir el archivo de entrada\n");
        return -1;
    }
  }
  else
    input_file = stdin;

  FILE* output_file;
  if(output_fileName != NULL)
  {
    if( (output_file = fopen(output_fileName, "wt")) == NULL)
    {
        fprintf(stderr,"No se pudo abrir el archivo de salida\n");
        return -1;
    }
  }
  else
    output_file = stdout;

  linuxToWindows(input_file, output_file);

  fclose(input_file);
  fclose(output_file);
  return 0;
}
\end{verbatim}

\subsection{dos2unix}
\begin{verbatim}
#include <stdio.h>
#include <stdlib.h>
#include <string.h>

#define MAX_NAME 100

const char CR = '\r';
const char LF = '\n';

int mostrarMensajeErrorParametrosInvalidos()
{
  fprintf(stderr, "Los parámetros ingresados no son válidos.\n");
  return -1;
}

void windowsToLinux(FILE* input_file, FILE* output_file)
{
	int c, d;

	while((c = fgetc(input_file)) != -1)
	{
		if ( c == CR )
		{
      		if((d = fgetc(input_file)) == -1)
      		{
        		fputc(CR, output_file);
        		break;
      		}
      		else
      		{
        		if( d == LF )
        		{
          			fputc(LF, output_file);
        		}
		        else
		        {
		          	fputc(CR, output_file);
		          	fputc(d, output_file);
		        }
      		}
    	}
    	else
    	{
      		fputc(c, output_file);
    	}

	}
}


int main(int argc, char** argv)
{
  // Inicializo los valores por defecto. No se ingresaron archivos y estos son stdin y stdout
  int seIngresoArchivoDeEntrada = 0;
  int seIngresoArchivoDeSalida = 0;
  char *input_fileName = NULL;
  char *output_fileName = NULL;

  /* No puedo recibir más de 5 parámetros. Este es el máximo esperado. Por otro lado, puedo
   recibir 1 parámetro (el nombre del programa), 3 parámetros (se especifica archivo de
   entrada o de salida) y 5 parámetros (se especifican ambos). Además argc siempre es mayor
   o igual a 1          */

  if (argc > 5 || argc == 4 || argc == 2) {
    return mostrarMensajeErrorParametrosInvalidos();
  }

  // Caso de recepción de al menos un archivo por parametro
  if (argc >= 3) {
    if (strcmp(argv[1], "-i") == 0) {
      if (strcmp(argv[2], "-") != 0)
      {
        input_fileName = argv[2];
        seIngresoArchivoDeEntrada = 1;
      }
    }
    else if (strcmp(argv[1], "-o") == 0) {
      if (strcmp(argv[2], "-") != 0)
      {
        output_fileName = argv[2];
        seIngresoArchivoDeSalida = 1;
      }
    }
    else {
      // Parámetro invalido
      return mostrarMensajeErrorParametrosInvalidos();
    }
    // Caso de recepción de dos archivos por parámetro
    if (argc > 4) {
      if (seIngresoArchivoDeEntrada) {
        if (strcmp(argv[3], "-o") == 0) {
          if (strcmp(argv[4], "-") != 0)
          {
            output_fileName = argv[4];
            seIngresoArchivoDeSalida = 1;
          }
        }
        else {
          // Si ya se ingreso el parametro "-i" el único válido que queda es "-o"
          return mostrarMensajeErrorParametrosInvalidos();
        }
      }
      else if (seIngresoArchivoDeSalida) {
        if (strcmp(argv[3], "-i") == 0) {
          if (strcmp(argv[4], "-") != 0)
          {
            input_fileName = argv[4];
            seIngresoArchivoDeEntrada = 1;
          }
        }
        else {
          // Si ya se ingreso el parametro "-o" el único válido que queda es "-i"
          return mostrarMensajeErrorParametrosInvalidos();
        }
      }
    }
  }

  FILE* input_file;
  if (input_fileName != NULL)
  {
    if( (input_file = fopen(input_fileName, "rt")) == NULL)
    {
        fprintf(stderr,"No se pudo abrir el archivo de entrada\n");
        return -1;
    }
  }
  else
    input_file = stdin;

  FILE* output_file;
  if(output_fileName != NULL)
  {
    if( (output_file = fopen(output_fileName, "wt")) == NULL)
    {
        fprintf(stderr,"No se pudo abrir el archivo de salida\n");
        return -1;
    }
  }
  else
    output_file = stdout;

  windowsToLinux(input_file, output_file);

  fclose(input_file);
  fclose(output_file);
  return 0;
}
\end{verbatim}


\section{C\'odigo MIPS}
\subsection{unix2dos}
\begin{verbatim}
		.file	1 "unix2dos.c"
	.section .mdebug.abi32
	.previous
	.abicalls
	.globl	CR
	.rdata
	.type	CR, @object
	.size	CR, 1
CR:
	.byte	13
	.globl	LF
	.type	LF, @object
	.size	LF, 1
LF:
	.byte	10
	.align	2
$LC0:
	.ascii	"Los par\303\241metros ingresados no son v\303\241lidos.\n"
	.ascii	"\000"
	.text
	.align	2
	.globl	mostrarMensajeErrorParametrosInvalidos
	.ent	mostrarMensajeErrorParametrosInvalidos
mostrarMensajeErrorParametrosInvalidos:
	.frame	$fp,40,$ra		# vars= 0, regs= 3/0, args= 16, extra= 8
	.mask	0xd0000000,-8
	.fmask	0x00000000,0
	.set	noreorder
	.cpload	$t9
	.set	reorder
	subu	$sp,$sp,40
	.cprestore 16
	sw	$ra,32($sp)
	sw	$fp,28($sp)
	sw	$gp,24($sp)
	move	$fp,$sp
	la	$a0,__sF+176
	la	$a1,$LC0
	la	$t9,fprintf
	jal	$ra,$t9
	li	$v0,-1			# 0xffffffffffffffff
	move	$sp,$fp
	lw	$ra,32($sp)
	lw	$fp,28($sp)
	addu	$sp,$sp,40
	j	$ra
	.end	mostrarMensajeErrorParametrosInvalidos
	.size	mostrarMensajeErrorParametrosInvalidos, .-mostrarMensajeErrorParametrosInvalidos
	.align	2
	.globl	linuxToWindows
	.ent	linuxToWindows
linuxToWindows:
	.frame	$fp,48,$ra		# vars= 8, regs= 3/0, args= 16, extra= 8
	.mask	0xd0000000,-8
	.fmask	0x00000000,0
	.set	noreorder
	.cpload	$t9
	.set	reorder
	subu	$sp,$sp,48
	.cprestore 16
	sw	$ra,40($sp)
	sw	$fp,36($sp)
	sw	$gp,32($sp)
	move	$fp,$sp
	sw	$a0,48($fp)
	sw	$a1,52($fp)
$L19:
	lw	$a0,48($fp)
	la	$t9,fgetc
	jal	$ra,$t9
	sw	$v0,24($fp)
	lw	$v1,24($fp)
	li	$v0,-1			# 0xffffffffffffffff
	bne	$v1,$v0,$L21
	b	$L18
$L21:
	lb	$v1,LF
	lw	$v0,24($fp)
	bne	$v0,$v1,$L22
	lb	$v0,CR
	move	$a0,$v0
	lw	$a1,52($fp)
	la	$t9,fputc
	jal	$ra,$t9
	lb	$v0,LF
	move	$a0,$v0
	lw	$a1,52($fp)
	la	$t9,fputc
	jal	$ra,$t9
	b	$L19
$L22:
	lw	$a0,24($fp)
	lw	$a1,52($fp)
	la	$t9,fputc
	jal	$ra,$t9
	b	$L19
$L18:
	move	$sp,$fp
	lw	$ra,40($sp)
	lw	$fp,36($sp)
	addu	$sp,$sp,48
	j	$ra
	.end	linuxToWindows
	.size	linuxToWindows, .-linuxToWindows
	.rdata
	.align	2
$LC1:
	.ascii	"-i\000"
	.align	2
$LC2:
	.ascii	"-\000"
	.align	2
$LC3:
	.ascii	"-o\000"
	.align	2
$LC4:
	.ascii	"rt\000"
	.align	2
$LC5:
	.ascii	"No se pudo abrir el archivo de entrada\n\000"
	.align	2
$LC6:
	.ascii	"wt\000"
	.align	2
$LC7:
	.ascii	"No se pudo abrir el archivo de salida\n\000"
	.text
	.align	2
	.globl	main
	.ent	main
main:
	.frame	$fp,72,$ra		# vars= 32, regs= 3/0, args= 16, extra= 8
	.mask	0xd0000000,-8
	.fmask	0x00000000,0
	.set	noreorder
	.cpload	$t9
	.set	reorder
	subu	$sp,$sp,72
	.cprestore 16
	sw	$ra,64($sp)
	sw	$fp,60($sp)
	sw	$gp,56($sp)
	move	$fp,$sp
	sw	$a0,72($fp)
	sw	$a1,76($fp)
	sw	$zero,24($fp)
	sw	$zero,28($fp)
	sw	$zero,32($fp)
	sw	$zero,36($fp)
	lw	$v0,72($fp)
	slt	$v0,$v0,6
	beq	$v0,$zero,$L26
	lw	$v1,72($fp)
	li	$v0,4			# 0x4
	beq	$v1,$v0,$L26
	lw	$v1,72($fp)
	li	$v0,2			# 0x2
	beq	$v1,$v0,$L26
	b	$L25
$L26:
	la	$t9,mostrarMensajeErrorParametrosInvalidos
	jal	$ra,$t9
	sw	$v0,48($fp)
	b	$L24
$L25:
	lw	$v0,72($fp)
	slt	$v0,$v0,3
	bne	$v0,$zero,$L27
	lw	$v0,76($fp)
	addu	$v0,$v0,4
	lw	$a0,0($v0)
	la	$a1,$LC1
	la	$t9,strcmp
	jal	$ra,$t9
	bne	$v0,$zero,$L28
	lw	$v0,76($fp)
	addu	$v0,$v0,8
	lw	$a0,0($v0)
	la	$a1,$LC2
	la	$t9,strcmp
	jal	$ra,$t9
	beq	$v0,$zero,$L30
	lw	$v0,76($fp)
	addu	$v0,$v0,8
	lw	$v0,0($v0)
	sw	$v0,32($fp)
	li	$v0,1			# 0x1
	sw	$v0,24($fp)
	b	$L30
$L28:
	lw	$v0,76($fp)
	addu	$v0,$v0,4
	lw	$a0,0($v0)
	la	$a1,$LC3
	la	$t9,strcmp
	jal	$ra,$t9
	bne	$v0,$zero,$L31
	lw	$v0,76($fp)
	addu	$v0,$v0,8
	lw	$a0,0($v0)
	la	$a1,$LC2
	la	$t9,strcmp
	jal	$ra,$t9
	beq	$v0,$zero,$L30
	lw	$v0,76($fp)
	addu	$v0,$v0,8
	lw	$v0,0($v0)
	sw	$v0,36($fp)
	li	$v0,1			# 0x1
	sw	$v0,28($fp)
	b	$L30
$L31:
	la	$t9,mostrarMensajeErrorParametrosInvalidos
	jal	$ra,$t9
	sw	$v0,48($fp)
	b	$L24
$L30:
	lw	$v0,72($fp)
	slt	$v0,$v0,5
	bne	$v0,$zero,$L27
	lw	$v0,24($fp)
	beq	$v0,$zero,$L35
	lw	$v0,76($fp)
	addu	$v0,$v0,12
	lw	$a0,0($v0)
	la	$a1,$LC3
	la	$t9,strcmp
	jal	$ra,$t9
	bne	$v0,$zero,$L36
	lw	$v0,76($fp)
	addu	$v0,$v0,16
	lw	$a0,0($v0)
	la	$a1,$LC2
	la	$t9,strcmp
	jal	$ra,$t9
	beq	$v0,$zero,$L27
	lw	$v0,76($fp)
	addu	$v0,$v0,16
	lw	$v0,0($v0)
	sw	$v0,36($fp)
	li	$v0,1			# 0x1
	sw	$v0,28($fp)
	b	$L27
$L36:
	la	$t9,mostrarMensajeErrorParametrosInvalidos
	jal	$ra,$t9
	sw	$v0,48($fp)
	b	$L24
$L35:
	lw	$v0,28($fp)
	beq	$v0,$zero,$L27
	lw	$v0,76($fp)
	addu	$v0,$v0,12
	lw	$a0,0($v0)
	la	$a1,$LC1
	la	$t9,strcmp
	jal	$ra,$t9
	bne	$v0,$zero,$L41
	lw	$v0,76($fp)
	addu	$v0,$v0,16
	lw	$a0,0($v0)
	la	$a1,$LC2
	la	$t9,strcmp
	jal	$ra,$t9
	beq	$v0,$zero,$L27
	lw	$v0,76($fp)
	addu	$v0,$v0,16
	lw	$v0,0($v0)
	sw	$v0,32($fp)
	li	$v0,1			# 0x1
	sw	$v0,24($fp)
	b	$L27
$L41:
	la	$t9,mostrarMensajeErrorParametrosInvalidos
	jal	$ra,$t9
	sw	$v0,48($fp)
	b	$L24
$L27:
	lw	$v0,32($fp)
	beq	$v0,$zero,$L44
	lw	$a0,32($fp)
	la	$a1,$LC4
	la	$t9,fopen
	jal	$ra,$t9
	sw	$v0,40($fp)
	lw	$v0,40($fp)
	bne	$v0,$zero,$L46
	la	$a0,__sF+176
	la	$a1,$LC5
	la	$t9,fprintf
	jal	$ra,$t9
	li	$v0,-1			# 0xffffffffffffffff
	sw	$v0,48($fp)
	b	$L24
$L44:
	la	$v0,__sF
	sw	$v0,40($fp)
$L46:
	lw	$v0,36($fp)
	beq	$v0,$zero,$L47
	lw	$a0,36($fp)
	la	$a1,$LC6
	la	$t9,fopen
	jal	$ra,$t9
	sw	$v0,44($fp)
	lw	$v0,44($fp)
	bne	$v0,$zero,$L49
	la	$a0,__sF+176
	la	$a1,$LC7
	la	$t9,fprintf
	jal	$ra,$t9
	li	$v0,-1			# 0xffffffffffffffff
	sw	$v0,48($fp)
	b	$L24
$L47:
	la	$v0,__sF+88
	sw	$v0,44($fp)
$L49:
	lw	$a0,40($fp)
	lw	$a1,44($fp)
	la	$t9,linuxToWindows
	jal	$ra,$t9
	lw	$a0,40($fp)
	la	$t9,fclose
	jal	$ra,$t9
	lw	$a0,44($fp)
	la	$t9,fclose
	jal	$ra,$t9
	sw	$zero,48($fp)
$L24:
	lw	$v0,48($fp)
	move	$sp,$fp
	lw	$ra,64($sp)
	lw	$fp,60($sp)
	addu	$sp,$sp,72
	j	$ra
	.end	main
	.size	main, .-main
	.ident	"GCC: (GNU) 3.3.3 (NetBSD nb3 20040520)"
\end{verbatim}

\subsection{dos2unix}
\begin{verbatim}
		.file	1 "dos2unix.c"
	.section .mdebug.abi32
	.previous
	.abicalls
	.globl	CR
	.rdata
	.type	CR, @object
	.size	CR, 1
CR:
	.byte	13
	.globl	LF
	.type	LF, @object
	.size	LF, 1
LF:
	.byte	10
	.align	2
$LC0:
	.ascii	"Los par\303\241metros ingresados no son v\303\241lidos.\n"
	.ascii	"\000"
	.text
	.align	2
	.globl	mostrarMensajeErrorParametrosInvalidos
	.ent	mostrarMensajeErrorParametrosInvalidos
mostrarMensajeErrorParametrosInvalidos:
	.frame	$fp,40,$ra		# vars= 0, regs= 3/0, args= 16, extra= 8
	.mask	0xd0000000,-8
	.fmask	0x00000000,0
	.set	noreorder
	.cpload	$t9
	.set	reorder
	subu	$sp,$sp,40
	.cprestore 16
	sw	$ra,32($sp)
	sw	$fp,28($sp)
	sw	$gp,24($sp)
	move	$fp,$sp
	la	$a0,__sF+176
	la	$a1,$LC0
	la	$t9,fprintf
	jal	$ra,$t9
	li	$v0,-1			# 0xffffffffffffffff
	move	$sp,$fp
	lw	$ra,32($sp)
	lw	$fp,28($sp)
	addu	$sp,$sp,40
	j	$ra
	.end	mostrarMensajeErrorParametrosInvalidos
	.size	mostrarMensajeErrorParametrosInvalidos, .-mostrarMensajeErrorParametrosInvalidos
	.align	2
	.globl	windowsToLinux
	.ent	windowsToLinux
windowsToLinux:
	.frame	$fp,48,$ra		# vars= 8, regs= 3/0, args= 16, extra= 8
	.mask	0xd0000000,-8
	.fmask	0x00000000,0
	.set	noreorder
	.cpload	$t9
	.set	reorder
	subu	$sp,$sp,48
	.cprestore 16
	sw	$ra,40($sp)
	sw	$fp,36($sp)
	sw	$gp,32($sp)
	move	$fp,$sp
	sw	$a0,48($fp)
	sw	$a1,52($fp)
$L19:
	lw	$a0,48($fp)
	la	$t9,fgetc
	jal	$ra,$t9
	sw	$v0,24($fp)
	lw	$v1,24($fp)
	li	$v0,-1			# 0xffffffffffffffff
	bne	$v1,$v0,$L21
	b	$L18
$L21:
	lb	$v1,CR
	lw	$v0,24($fp)
	bne	$v0,$v1,$L22
	lw	$a0,48($fp)
	la	$t9,fgetc
	jal	$ra,$t9
	sw	$v0,28($fp)
	lw	$v1,28($fp)
	li	$v0,-1			# 0xffffffffffffffff
	bne	$v1,$v0,$L23
	lb	$v0,CR
	move	$a0,$v0
	lw	$a1,52($fp)
	la	$t9,fputc
	jal	$ra,$t9
	b	$L18
$L23:
	lb	$v1,LF
	lw	$v0,28($fp)
	bne	$v0,$v1,$L25
	lb	$v0,LF
	move	$a0,$v0
	lw	$a1,52($fp)
	la	$t9,fputc
	jal	$ra,$t9
	b	$L19
$L25:
	lb	$v0,CR
	move	$a0,$v0
	lw	$a1,52($fp)
	la	$t9,fputc
	jal	$ra,$t9
	lw	$a0,28($fp)
	lw	$a1,52($fp)
	la	$t9,fputc
	jal	$ra,$t9
	b	$L19
$L22:
	lw	$a0,24($fp)
	lw	$a1,52($fp)
	la	$t9,fputc
	jal	$ra,$t9
	b	$L19
$L18:
	move	$sp,$fp
	lw	$ra,40($sp)
	lw	$fp,36($sp)
	addu	$sp,$sp,48
	j	$ra
	.end	windowsToLinux
	.size	windowsToLinux, .-windowsToLinux
	.rdata
	.align	2
$LC1:
	.ascii	"-i\000"
	.align	2
$LC2:
	.ascii	"-\000"
	.align	2
$LC3:
	.ascii	"-o\000"
	.align	2
$LC4:
	.ascii	"rt\000"
	.align	2
$LC5:
	.ascii	"No se pudo abrir el archivo de entrada\n\000"
	.align	2
$LC6:
	.ascii	"wt\000"
	.align	2
$LC7:
	.ascii	"No se pudo abrir el archivo de salida\n\000"
	.text
	.align	2
	.globl	main
	.ent	main
main:
	.frame	$fp,72,$ra		# vars= 32, regs= 3/0, args= 16, extra= 8
	.mask	0xd0000000,-8
	.fmask	0x00000000,0
	.set	noreorder
	.cpload	$t9
	.set	reorder
	subu	$sp,$sp,72
	.cprestore 16
	sw	$ra,64($sp)
	sw	$fp,60($sp)
	sw	$gp,56($sp)
	move	$fp,$sp
	sw	$a0,72($fp)
	sw	$a1,76($fp)
	sw	$zero,24($fp)
	sw	$zero,28($fp)
	sw	$zero,32($fp)
	sw	$zero,36($fp)
	lw	$v0,72($fp)
	slt	$v0,$v0,6
	beq	$v0,$zero,$L30
	lw	$v1,72($fp)
	li	$v0,4			# 0x4
	beq	$v1,$v0,$L30
	lw	$v1,72($fp)
	li	$v0,2			# 0x2
	beq	$v1,$v0,$L30
	b	$L29
$L30:
	la	$t9,mostrarMensajeErrorParametrosInvalidos
	jal	$ra,$t9
	sw	$v0,48($fp)
	b	$L28
$L29:
	lw	$v0,72($fp)
	slt	$v0,$v0,3
	bne	$v0,$zero,$L31
	lw	$v0,76($fp)
	addu	$v0,$v0,4
	lw	$a0,0($v0)
	la	$a1,$LC1
	la	$t9,strcmp
	jal	$ra,$t9
	bne	$v0,$zero,$L32
	lw	$v0,76($fp)
	addu	$v0,$v0,8
	lw	$a0,0($v0)
	la	$a1,$LC2
	la	$t9,strcmp
	jal	$ra,$t9
	beq	$v0,$zero,$L34
	lw	$v0,76($fp)
	addu	$v0,$v0,8
	lw	$v0,0($v0)
	sw	$v0,32($fp)
	li	$v0,1			# 0x1
	sw	$v0,24($fp)
	b	$L34
$L32:
	lw	$v0,76($fp)
	addu	$v0,$v0,4
	lw	$a0,0($v0)
	la	$a1,$LC3
	la	$t9,strcmp
	jal	$ra,$t9
	bne	$v0,$zero,$L35
	lw	$v0,76($fp)
	addu	$v0,$v0,8
	lw	$a0,0($v0)
	la	$a1,$LC2
	la	$t9,strcmp
	jal	$ra,$t9
	beq	$v0,$zero,$L34
	lw	$v0,76($fp)
	addu	$v0,$v0,8
	lw	$v0,0($v0)
	sw	$v0,36($fp)
	li	$v0,1			# 0x1
	sw	$v0,28($fp)
	b	$L34
$L35:
	la	$t9,mostrarMensajeErrorParametrosInvalidos
	jal	$ra,$t9
	sw	$v0,48($fp)
	b	$L28
$L34:
	lw	$v0,72($fp)
	slt	$v0,$v0,5
	bne	$v0,$zero,$L31
	lw	$v0,24($fp)
	beq	$v0,$zero,$L39
	lw	$v0,76($fp)
	addu	$v0,$v0,12
	lw	$a0,0($v0)
	la	$a1,$LC3
	la	$t9,strcmp
	jal	$ra,$t9
	bne	$v0,$zero,$L40
	lw	$v0,76($fp)
	addu	$v0,$v0,16
	lw	$a0,0($v0)
	la	$a1,$LC2
	la	$t9,strcmp
	jal	$ra,$t9
	beq	$v0,$zero,$L31
	lw	$v0,76($fp)
	addu	$v0,$v0,16
	lw	$v0,0($v0)
	sw	$v0,36($fp)
	li	$v0,1			# 0x1
	sw	$v0,28($fp)
	b	$L31
$L40:
	la	$t9,mostrarMensajeErrorParametrosInvalidos
	jal	$ra,$t9
	sw	$v0,48($fp)
	b	$L28
$L39:
	lw	$v0,28($fp)
	beq	$v0,$zero,$L31
	lw	$v0,76($fp)
	addu	$v0,$v0,12
	lw	$a0,0($v0)
	la	$a1,$LC1
	la	$t9,strcmp
	jal	$ra,$t9
	bne	$v0,$zero,$L45
	lw	$v0,76($fp)
	addu	$v0,$v0,16
	lw	$a0,0($v0)
	la	$a1,$LC2
	la	$t9,strcmp
	jal	$ra,$t9
	beq	$v0,$zero,$L31
	lw	$v0,76($fp)
	addu	$v0,$v0,16
	lw	$v0,0($v0)
	sw	$v0,32($fp)
	li	$v0,1			# 0x1
	sw	$v0,24($fp)
	b	$L31
$L45:
	la	$t9,mostrarMensajeErrorParametrosInvalidos
	jal	$ra,$t9
	sw	$v0,48($fp)
	b	$L28
$L31:
	lw	$v0,32($fp)
	beq	$v0,$zero,$L48
	lw	$a0,32($fp)
	la	$a1,$LC4
	la	$t9,fopen
	jal	$ra,$t9
	sw	$v0,40($fp)
	lw	$v0,40($fp)
	bne	$v0,$zero,$L50
	la	$a0,__sF+176
	la	$a1,$LC5
	la	$t9,fprintf
	jal	$ra,$t9
	li	$v0,-1			# 0xffffffffffffffff
	sw	$v0,48($fp)
	b	$L28
$L48:
	la	$v0,__sF
	sw	$v0,40($fp)
$L50:
	lw	$v0,36($fp)
	beq	$v0,$zero,$L51
	lw	$a0,36($fp)
	la	$a1,$LC6
	la	$t9,fopen
	jal	$ra,$t9
	sw	$v0,44($fp)
	lw	$v0,44($fp)
	bne	$v0,$zero,$L53
	la	$a0,__sF+176
	la	$a1,$LC7
	la	$t9,fprintf
	jal	$ra,$t9
	li	$v0,-1			# 0xffffffffffffffff
	sw	$v0,48($fp)
	b	$L28
$L51:
	la	$v0,__sF+88
	sw	$v0,44($fp)
$L53:
	lw	$a0,40($fp)
	lw	$a1,44($fp)
	la	$t9,windowsToLinux
	jal	$ra,$t9
	lw	$a0,40($fp)
	la	$t9,fclose
	jal	$ra,$t9
	lw	$a0,44($fp)
	la	$t9,fclose
	jal	$ra,$t9
	sw	$zero,48($fp)
$L28:
	lw	$v0,48($fp)
	move	$sp,$fp
	lw	$ra,64($sp)
	lw	$fp,60($sp)
	addu	$sp,$sp,72
	j	$ra
	.end	main
	.size	main, .-main
	.ident	"GCC: (GNU) 3.3.3 (NetBSD nb3 20040520)"
\end{verbatim}

\section{Conclusiones}
Como conclusi\'on notamos que la mayor dificultad para desarrollar los 2 programas pedidos, surgi\'o para la lectura de los archivos caracter por caracter. Como se mencion\'o anteriormente, primero usamos como condici\'on EOF. Esto nos gener\'o salidas distintas a las esperadas. Por otro lado, cuando usamos la funci\'on fgetc tomamos erroneamente que el resultado que esta funci\'on devolv\'ia era un char cuando en realidad era un int. Esto tambi\'en gener\'o que para casos m\'as especiales, las salidas no fueran las esperadas.

\end{document}



