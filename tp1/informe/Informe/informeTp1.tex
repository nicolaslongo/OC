\documentclass[a4paper,11pt]{article}

\usepackage[spanish]{babel}
\usepackage{graphicx}
\usepackage[ansinew]{inputenc}
\usepackage[utf8]{inputenx}

\title{		\textbf{Informe TP1}}

\author{	Cotarelo Rodrigo, \textit{Padrón Nro. 98577}                     \\
            \texttt{ cotarelorodrigo@gmail.com }                                              \\[2.5ex]
            Etchegaray Rodrigo, \textit{Padrón Nro. 96856}                     \\
            \texttt{ rorroeche@gmail.com }                                              \\[2.5ex]
			Longo Nicolás, \textit{Padrón Nro. 98271}                    
\\
            \texttt{ longo.gnr@hotmail.com }                                              \\[2.5ex]
            \normalsize{1er. Cuatrimestre de 2019}                                      \\
            \normalsize{66.20 Organización de Computadoras  $-$ Práctica Martes}  \\
            \normalsize{Facultad de Ingeniería, Universidad de Buenos Aires}            \\
       }
\date{7/5/19}


\begin{document}

\maketitle
\thispagestyle{empty}   % quita el número en la primer página
\newpage

\section{Enunciado}

\begin{enumerate}
\item Objetivos \\
Familiarizarse con el conjunto de instrucciones MIPS y el concepto de ABI,
extendiendo un programa que resuelva el problema descripto en la secci\'on 4.

\item Alcance \\
Este trabajo p\'actico es de elaboraci\'on grupal, evaluaci\'on individual, y de
car\'acter obligatorio para todos alumnos del curso.

\item Requisitos \\
El informe deber\'a ser entregado personalmente, por escrito, en la fecha
estipulada, con una car\'atula que contenga los datos completos de todos los
integrantes.
Adem\'as, es necesario que el trabajo pr\'actico incluya (entre otras cosas,
ver secci\'on 6), la presentaci\'on de los resultados obtenidos, explicando, cuando
corresponda, con fundamentos reales, las causas o razones de cada caso.

\item Descripci\'on \\
En este trabajo, vamos a implementar dos programas en assembly MIPS,
que nos permitan convertir informaci\'on almacenada en streams de texto, de
acuerdo a lo descripto en el trabajo anterior [1]: unix2dos, para transformar
archivos de texto de UNIX a DOS; y dos2unix para realizar la operaci\'on inversa.
Es decir, siguiendo la l\'inea de trabajo de los pr\'acticos anteriores, buscamos
generar dos binarios diferentes en vez de un \'unico programa que resuelva
ambos problemas. Adem\'as, por tratarse de un TP orientado a practicar el
conjunto de instrucciones, ambos programas deber\'an escritos completamente
en assembly, en forma manual.

\item Interfaz \\
Por defecto, ambos programas operan exclusivamente por stdin y stdout.
Al igual que en el trabajo anterior, al finalizar, estos programas retornar\'an
un valor nulo en caso de no detectar ning\'un problema; y, en caso contrario,
finalizar\'an con c\'odigo no nulo (por ejemplo 1).


\item Ejemplos \\
Cuando el documento de entrada es vac\'io, la salida deber\'ia serlo tambi\'en:
\begin{verbatim}
$ unix2dos </dev/null | wc -c
0
$ dos2unix </dev/null | wc -c
0
\end{verbatim}
Tambi\'en se aplican todos los casos de prueba definidos para el TP anterior.
Por ejemplo:
\begin{verbatim}
$ od -t c /tmp/dos.txt
0000000 U n o \r \n D o s \r \n T r e s \r \n
0000020
$ dos2unix -i /tmp/dos.txt -o - | od -t c
0000000 U n o \n D o s \n T r e s \n
0000015
\end{verbatim}
Por \'ultimo, podemos usar el programa que sigue para generar secuencias de
datos aleatorias, pasarlas a hexa, reconvertirlas al dominio original, y verificar
que la operaci\'on no haya alterado la informaci\'on:
\begin{verbatim}
$ n=0; while :; do
> n="‘expr $n + 1‘";
> head -c 100 </dev/urandom >/tmp/test.$n.u;
> unix2dos </tmp/test.$n.u >/tmp/test.$n.d || break;
> dos2unix </tmp/test.$n.d >/tmp/test.$n.2.u || break;
> diff -q /tmp/test.$n.u /tmp/test.$n.2.u || break;
> rm -f /tmp/test.$n.*;
> echo Ok: $n;
> done; echo Error: $n
Ok: 1
Ok: 2
Ok: 3
...
\end{verbatim}

\item Implementaci\'on \\
El programa a implementar deber\'a satisfacer algunos requerimientos m\'inimos,
que detallamos a continuaci\'on:


\begin{enumerate}
\item ABI \\
Ser\'a necesario que el c\'odigo presentado utilice la ABI explicada en la clase:
los argumentos correspondientes a los registros $a0-$a3 ser\'an almacenados
por el callee, siempre, en los 16 bytes dedicados de la secci\'on “function call
argument area” [2].

\item Casos de prueba\\
Es necesario que la implementaci\'on propuesta pase todos los casos incluidos
tanto en este enunciado como en el conjunto pruebas suministrado en el
informe del trabajo, los cuales deber\'an estar debidamente documentados y
justificados.

\item Documentaci\'on \\
El informe deber\'a incluir una descripci\'on detallada de las t\'ecnicas y procesos
de medici\'on empleados, y de todos los pasos involucrados en el desarrollo
del mismo, ya que forman parte de los objetivos principales del trabajo.
\end{enumerate}

\item Informe\\
El informe deber\'a incluir:
\begin{itemize}
\item Este enunciado;

\item Documentaci\'on relevante al disenio, implementaci\'on y medici\'on de las
caracter\'isticas del programa;

\item Las corridas de prueba, (secci\'on 5.2) con los comentarios pertinentes;

\item El c\'odigo fuente completo, en dos formatos: digitalizado1 e impreso en
papel. 
\end{itemize}

\item Fechas \\
Fecha de vencimiento: 7/5.

\item Referencias
Enunciado del primer trabajo pr\'actico, primer cuatrimestre de 2019.
http://groups.yahoo.com/groups/orga-comp/.

\end{enumerate}
\newpage

\section{Implementaci\'on}
Se implementaron dos funciones para poder convertir un archivo de DOS a UNIX y viceversa. La implementaci\'on se basó en un ciclo while que recorre el archivo byte por byte mientras consulta si está parado sobre un '$\backslash$r' o '$\backslash$n'. En caso de ser alguno de estos caracteres, se efectua la modificación correspondiente y sino, simplemente se escribe el mismo caracter en la salida.
Para lograr esto, decidimos dividir los programas en pequeñas funciones.Esto nos permiti\'o por un lado reducir la complejidad e ir teniendo seguridad de que las cosas que desarroll\'abamos funcionaban de la manera correcta. Y por otro lado, reutilizar funciones que aplicaban tanto a unix2dos como a dos2unix.

\subsection{Prototipos}
\begin{itemize}
\item getch: recibe como \'unico par\'ametro el buffer. La lectura siempre es de a 1 byte y por stdin. Como resultado devuelve el caracter le\'ido por stdin y en caso de ser EOF devuelve -1.

\item putch: recibe como \'unico par\'ametro el byte a escribir. La escritura siempre es de a 1 byte y por stdout. Como resultado escribe el byte que recibe por stdout.

\item isLF: recibe como \'unico par\'ametro el byte a analizar. Devuelve 1 si el caracter es LF y 0 si no lo es.

\item isCR: recibe como \'unico par\'ametro el byte a analizar. Devuelve 1 si el caracter es CR y 0 si no lo es.
\end{itemize}

\section{Generaci\'on de binarios}
Para la generaci\'on de los binarios es necesario utilizar la consola de Linux corriendo sobre la imagen de NetBSD que nos brinda la cátedra. Se levanta el túnel, se copian los archivos desde el HostOS hacia el GuestOS, y finalmente desde este último se realiza la compilación. Para ello se debe utilizar gcc de la siguiente forma (usaremos de ejemplo el programa unix2dos, pero aplica para ambos programas):

\begin{itemize}
\item Abir la terminal y posicionarse en el directorio donde se encuentren los archivos con el c\'odigo fuente. Siguiendo la lógica de nuestro repositorio sería en \textit{.../tp1/src/}
\item \begin{verbatim}
gcc -Wall -O0 -o unix2dos getch.S putch.S isLF.S isCR.S unix2dos.S
\end{verbatim} 
\end{itemize}

\section{Utilización}
Para ejecutar los programas se debe utilizar la consola o shell del GuestOS (es decir, de nuestra instancia de Linux corriendo en NetBSD). Dependiendo de qué tipo de conversión de archivo se desee, se usará alguno de los dos programas. La ejecución de cualquiera de estos utiliza la entrada estándar (\textit{stdin}) para la lectura de los caracteres y la salida estándar (\textit{stdout}) para imprimir los resultados obtenidos.

\indent	
Por ejemplo, para convertir un archivo de DOS a UNIX se ejecuta: 

\begin{verbatim}
echo -e -n "Prueba que \r\nincluye algunos \r\nenters" | ./dos2unix 
\end{verbatim}

\section{Pruebas}
\subsection{unix2dos}
Primero probamos que con entrada vac\'ia, la salida deb\'ia serlo tambi\'en.
\begin{verbatim}
root@:~/tp1# ./unix2dos </dev/null | wc -c
       0
\end{verbatim}

Luego hicimos una prueba con contenido chequeando que la salida fuera la misma pero con la transformaci\'on correspondiente.
\begin{verbatim}
root@:~/tp1# echo -n -e "Uno \n Dos \n Tres \n" | od -t c
0000000    U   n   o      \n       D   o   s      \n       T   r   e   s
0000020       \n                                                        
0000022
root@:~/tp1# echo -n -e "Uno \n Dos \n Tres \n" |./unix2dos |od -t c
0000000    U   n   o      \r  \n       D   o   s      \r  \n       T   r
0000020    e   s      \r  \n                                            
0000025
\end{verbatim}

Por \'ultimo hicimos una prueba un poco m\'as compleja
\begin{verbatim}
root@:~/tp1# echo -n -e "Archivo mas \n lardo    
con varios espacios y varios  \n\n\n enters para tratart 
de marear \n al programa. Meto un \n\n enter al final \n" | od -t c
0000000    A   r   c   h   i   v   o       m   a   s      \n       l   a
0000020    r   d   o                           c   o   n       v   a   r
0000040    i   o   s       e   s   p   a   c   i   o   s       y       v
0000060    a   r   i   o   s          \n  \n  \n       e   n   t   e   r
0000100    s       p   a   r   a       t   r   a   t   a   r   t       d
0000120    e       m   a   r   e   a   r      \n       a   l       p   r
0000140    o   g   r   a   m   a   .       M   e   t   o       u   n    
0000160   \n  \n       e   n   t   e   r       a   l       f   i   n   a
0000200    l      \n                                                    
0000203
root@:~/tp1# echo -n -e "Archivo mas \n lardo 
     con varios espacios y varios  
     \n\n\n enters para tratart 
     de marear \n al programa. 
     Meto un \n\n enter al final \n" |./unix2dos |od -t c
0000000    A   r   c   h   i   v   o       m   a   s      \r  \n       l
0000020    a   r   d   o                           c   o   n       v   a
0000040    r   i   o   s       e   s   p   a   c   i   o   s       y    
0000060    v   a   r   i   o   s          \r  \n  \r  \n  \r  \n       e
0000100    n   t   e   r   s       p   a   r   a       t   r   a   t   a
0000120    r   t       d   e       m   a   r   e   a   r      \r  \n    
0000140    a   l       p   r   o   g   r   a   m   a   .       M   e   t
0000160    o       u   n      \r  \n  \r  \n       e   n   t   e   r    
0000200    a   l       f   i   n   a   l      \r  \n                    
0000213
\end{verbatim}

\subsection{dos2unix}
Primero probamos que con entrada vac\'ia, la salida deb\'ia serlo tambi\'en.
\begin{verbatim}
root@:~/tp1# ./dos2unix </dev/null | wc -c
       0
\end{verbatim}

Luego hicimos una prueba con contenido chequeando que la salida fuera la misma pero con la transformaci\'on correspondiente.
\begin{verbatim}
root@:~/tp1# echo -n -e "Uno \r\n Dos \r\n Tres \r\n" | od -t c
0000000    U   n   o      \r  \n       D   o   s      \r  \n       T   r
0000020    e   s      \r  \n                                            
0000025
root@:~/tp1# echo -n -e "Uno \r\n Dos \r\n Tres \r\n" |./dos2unix| od -t c
0000000    U   n   o      \n       D   o   s      \n       T   r   e   s
0000020       \n                                                        
0000022
\end{verbatim}

Por \'ultimo hicimos una prueba un poco m\'as compleja
\begin{verbatim}
root@:~/tp1# echo -n -e "Archivo mas \r\n lardo      
con varios espacios y varios  \r\n\r\n\r\n enters para
 tratart de marear \r\n al programa. 
 Meto un \r\n\r\n enter al final \r\n" | od -t c
0000000    A   r   c   h   i   v   o       m   a   s      \r  \n       l
0000020    a   r   d   o                           c   o   n       v   a
0000040    r   i   o   s       e   s   p   a   c   i   o   s       y    
0000060    v   a   r   i   o   s          \r  \n  \r  \n  \r  \n       e
0000100    n   t   e   r   s       p   a   r   a       t   r   a   t   a
0000120    r   t       d   e       m   a   r   e   a   r      \r  \n    
0000140    a   l       p   r   o   g   r   a   m   a   .       M   e   t
0000160    o       u   n      \r  \n  \r  \n       e   n   t   e   r    
0000200    a   l       f   i   n   a   l      \r  \n                    
0000213
root@:~/tp1# echo -n -e "Archivo mas \r\n lardo 
     con varios espacios y varios
       \r\n\r\n\r\n enters para
        tratart de marear \r\n al programa. 
        Meto un \r\n\r\n enter al final \r\n" |./dos2unix |od -t c
0000000    A   r   c   h   i   v   o       m   a   s      \n       l   a
0000020    r   d   o                           c   o   n       v   a   r
0000040    i   o   s       e   s   p   a   c   i   o   s       y       v
0000060    a   r   i   o   s          \n  \n  \n       e   n   t   e   r
0000100    s       p   a   r   a       t   r   a   t   a   r   t       d
0000120    e       m   a   r   e   a   r      \n       a   l       p   r
0000140    o   g   r   a   m   a   .       M   e   t   o       u   n    
0000160   \n  \n       e   n   t   e   r       a   l       f   i   n   a
0000200    l      \n                                                    
0000203
\end{verbatim}

{C\'odigo fuente}
\subsection{getch}
\begin{verbatim}
#include <mips/regdef.h>
#include <sys/syscall.h>
.text
.abicalls
.globl getch
.ent getch

getch:
	.frame $fp, 8, ra
	.set noreorder
	.cpload t9
	.set reorder

	subu sp, sp, 8 	# pido espacio para mi Stack Frame
	.cprestore 4		# salvo gp en 28
	sw $fp, 0(sp)		# salvo fp en 24
	#sw ra, 32(sp)		# salvo ra en 32
	move $fp, sp		# a partir de acá trabajo con fp

# me guardo los parámetros que no guardo la caller (por convención de ABI)
	sw a0, 8($fp)		# salvo el buffer adonde va lo leído (en este caso es un int)

# Se supone que el file descriptor es STDIN (0) y la cantidad de bytes es 1
# hago la lectura propiamente dicha:
_lectura:
	la a0, 0			# STDIN
	lw a1, 8($fp) 		# buffer de lo leido
	li a2, 1			# (1 acá es 1 byte)
	li v0, SYS_read
	syscall
	#move a1, v1

_comprobacion:
	bltz a3, _error		# si a3 es -1, falló la lectura del byte -> acá debería haber un catch del error, probablemente
	beqz v0, _eof			# (en a3 ya tengo un 0) si en v0 tengo un 0, entonces EOF
	li t0, 1
	beq t0, v0, _return		# esto ya es obvio, en realidad

_return:
# tengo en v0 la posición en memoria del char (leído) y
# me lo llevo con lb -> mucho muy importante, porque es un char
	lb v0, 8($fp)			# en v0 guardo el char (leído) que es lo que voy a devolver (o -1 si terminó)
# Devolvemos las cosas a su lugar
	#lw ra, 32(sp)
	lw $fp, 0(sp)
	lw gp, 4(sp)
	addu sp, sp, 8
	jr	ra

_eof:
	li t0, -1
	sw t0, 8($fp)
	b _return

_error:		# en caso de error, aborto el programa con código 1
	li a0, 1
	li v0, SYS_exit
	syscall


.end getch
\end{verbatim}

\subsection{putch}
\begin{verbatim}
#include <mips/regdef.h>
#include <sys/syscall.h>
.text
.abicalls
.globl putch
.ent putch

putch:
	.frame $fp, 8, ra
	.set noreorder
	.cpload t9
	.set reorder

	subu sp, sp, 8 	# pido espacio para mi Stack Frame
	.cprestore 4		# guardo gp en 28
	sw $fp, 0(sp)		# guardo fp en 24
	# sw ra, 32(sp)		# guardo ra en 32
	move $fp, sp		# a partir de acá trabajo con fp

# me guardo los parámetros que no guardo la caller (por convención de ABI)
# EL CHAR QUE LLEGA A ÉSTA ALTURA TIENE QUE HABER SIDO CARGADO CON lb !!!!!!!
	sw a0, 8($fp)		# guardo el char a escribir

_escritura:
	la a0, 1			# stdout
	lw t0, 8($fp)
	la a1, 0(t0)
	li a2, 1			# cantidad de bytes
	li v0, SYS_write
	syscall

	bnez a3, _escritura 		# Si da error intentamos de nuevo

_return:
	#lw ra, 32(sp)
	lw $fp, 0(sp)
	lw gp, 4(sp)
	addu sp, sp, 8
	jr ra

.end putch
\end{verbatim}

\subsection{isLF}
\begin{verbatim}
#include <mips/regdef.h>
#include <sys/syscall.h>
.text
.abicalls
.globl isLF
.ent isLF

isLF:
	.frame $fp, 40, ra
	.set noreorder
	.cpload t9
	.set reorder

	subu sp, sp, 40 	# pido espacio para mi Stack Frame
	.cprestore 28		# guardo gp en 20
	sw $fp, 24(sp)		# guardo fp en 16
	sw ra, 32(sp)		# guardo ra en 24
	move $fp, sp		# a partir de acá trabajo con fp

	sw a0, 40($fp)		# me guardo el char a comparar a 40 de fp, aunque sea innecesario
	lb t1, 0(a0)		# cargo el char en sí ES UN BYTE
	la t0, LF			
	lb t2, 0(t0)		# cargo el valor de LF según código ASCII

	beq t1, t2, _true
	b _false

_true:
	li t0, 1
	sb t0, 16($fp)
	b _end

_false:
	li v0, 0
	sb t0, 16($fp)

_end:
	# cargo en v0 el vaLor del booleano
	lb v0, 16($fp)
	lw ra, 32(sp)
	lw $fp, 24(sp)
	lw gp, 28(sp)
	addu sp, sp, 40
	jr	ra

.end isLF

.data

LF: .byte 10
\end{verbatim}

\subsection{isCR}
\begin{verbatim}
#include <mips/regdef.h>
#include <sys/syscall.h>
.text
.abicalls
.globl isCR
.ent isCR

isCR:
	.frame $fp, 40, ra
	.set noreorder
	.cpload t9
	.set reorder

	subu sp, sp, 40 	# pido espacio para mi Stack Frame
	.cprestore 28		# guardo gp en 20
	sw $fp, 24(sp)		# guardo fp en 16
	sw ra, 32(sp)		# guardo ra en 24
	move $fp, sp		# a partir de acá trabajo con fp

	sw a0, 40($fp)		# me guardo el char a comparar a 40 de fp, aunque sea innecesario
	lb t1, 0(a0)		# cargo el char en sí ES UN BYTE
	la t0, CR			
	lb t2, 0(t0)		# cargo el valor de CR según código ASCII

	beq t1, t2, _true
	b _false

_true:
	li t0, 1
	sb t0, 16($fp)
	b _end

_false:
	li v0, 0
	sb t0, 16($fp)

_end:
	# cargo en v0 el vaLor del booleano
	lb v0, 16($fp)
	lw ra, 32(sp)
	lw $fp, 24(sp)
	lw gp, 28(sp)
	addu sp, sp, 40
	jr	ra

.end isCR

.data

CR: .byte 13
\end{verbatim}

\subsection{unix2dos}
\begin{verbatim}
#include <mips/regdef.h>
#include <sys/syscall.h>
.text
.abicalls
.globl main
.ent main

main:
	.frame $fp, 32, ra
	.set noreorder
	.cpload t9
	.set reorder

	subu sp, sp, 32 	# pido espacio para mi Stack Frame
	.cprestore 20		# guardo gp en 20
	sw $fp, 16(sp)		# guardo fp en 16
	sw ra, 24(sp)		# guardo ra en 24
	move $fp, sp		# a partir de acá trabajo con fp

# hago una lectura:
_lectura:

	la a0, BUFFER
	jal getch

	# en v0 me devuelve el byte leído
	li t0, -1
	beq v0, t0, _eof

	# me fijo si tengo un LF
	la a0, BUFFER
	jal isLF

	li t0, 1
	beq v0, t0, _replace_LF

	la a0, BUFFER
	jal putch

	b _lectura

_eof:
_return:
	lw ra, 24(sp)
	lw $fp, 16(sp)
	lw gp, 20(sp)
	addu sp, sp, 32
	li v0, SYS_exit
	li a0, 0
	syscall

_replace_LF:
	la a0, CR
	jal putch

	la a0, BUFFER		# acá sigo teniendo LF
	jal putch

	b _lectura

.end main

.data
BUFFER: .space 1
CR: .byte 13
\end{verbatim}

\subsection{dos2unix}
\begin{verbatim}
#include <mips/regdef.h>
#include <sys/syscall.h>
.text #Lo que viene es codigo
.abicalls #Voy a usar la convencion abi
.globl main
.ent main


main:
	.frame $fp, 32, ra
	.set noreorder #RECETA magica no explicada
	.cpload t9 #RECETA magica no explicada
	.set reorder #RECETA magica no explicada

	subu sp, sp, 32 	# pido espacio para mi Stack Frame
	.cprestore 20		# guardo gp en 20
	sw $fp, 16(sp)		# guardo fp en 16
	sw ra, 24(sp)		# guardo ra en 32
	move $fp, sp		# a partir de acá trabajo con fp

# hago una lectura:
_lectura:

	# Cargo el primer argumento
	la a0, BUFFER

	# Llamo a getch
	jal getch # en v0 me devuelve el byte leído

	li t0, -1
	beq v0, t0, _eof

continue:
	# me fijo si tengo un CR
	la a0, BUFFER
	jal isCR

	li t0, 1
	beq v0, t0, _checkNextIsLF

	la a0, BUFFER
	jal putch

	b _lectura

_eof:
_return:
	lw ra, 24(sp)
	lw $fp, 16(sp)
	lw gp, 20(sp)
	addu sp, sp, 32
	li a0, 0
	li v0, SYS_exit
	syscall

_checkNextIsLF:
	# Llamo a getch pero antes, guardo el byte anterior en t3
	jal getch # en v0 me devuelve el byte leído

	li t0, -1
	beq v0, t0, _putCRandBreak

	# me fijo si tengo un LF
	la a0, BUFFER
	jal isLF

	li t0, 1
	beq v0, t0, _putLF

	la a0, CR
	jal putch

	b continue


_putCRandBreak:
	la a0, CR
	jal putch

	b _eof # break;

_putLF:
	la a0, LF
	jal putch

	b _lectura

.end main

.data
BUFFER: .space 1
CR: .byte 13
LF: .byte 10
\end{verbatim}

\section{Documentación pertinente a los stack frames}

El diagrama de stack frames por función se encuentra en el directorio fuente del trabajo práctico, en el archivo stackframes.txt (fue dejado ahí por una cuestión de formato y facilidad de lectura).

\section{Conclusiones}

Como este trabajo práctico fue una réplica del trabajo práctico anterior (TP0), migrando la funcionalidad desde código C a código Assembly (de MIPS32), una de las claves del mismo fue su desarrollo en pequeños pasos incrementales.

\indent
Utilizando los programas unix2dos.c y dos2unix.c que resultaron de la entrega del TP0, fuimos migrando la funcionalidad hacia MIPS empezando por las funciones getch (en getch.S) y putch (en putch.S). Estas reemplazaron las llamadas a getchar() y putchar() en las funciones de C. Luego, se implementó la detección de bytes que correspondieran a CR o LF: cada uno de estos programas se almaceno en isCR.S e isLF.S, correspondientemente. Una vez que pudimos identificarlos, procedimos a su correcto reemplazo.
Por último se implementó el ciclo while desde Assembly para finalmente desligarse por completo de la utilización de un main de C. Para ello, la función que implementó el while debió llamarse main y así GCC la seteó como punto de entrada del programa.

\indent
Algunas de las complicaciones más importantes que se presentaron fueron las de diferenciar cuándo lidiábamos con una dirección de memoria (lo que regularmente, y en más alto nivel, llamamos puntero) y cuándo con un dato en sí. De la misma manera, tuvimos que ser sumamente cuidadosos a la hora de manipular estos últimos conociendo su tamaño: si se está manejando un byte las operaciones son unas y si se está manejando un word, son otras distintas.

\indent
Finalmente, es necesario destacar la relevancia de contar con un programa como gdb que permite realizar tareas de Debugging de manera muy completa y brindando muchísimas facilidades.


\end{document}
