\documentclass[a4paper,11pt]{article}

\usepackage[spanish]{babel}
\usepackage{graphicx}
\usepackage[ansinew]{inputenc}
\usepackage[utf8]{inputenx}

\title{		\textbf{Informe TP1}}

\author{	Cotarelo Rodrigo, \textit{Padrón Nro. 98577}                     \\
            \texttt{ cotarelorodrigo@gmail.com }                                              \\[2.5ex]
            Etchegaray Rodrigo, \textit{Padrón Nro. 96856}                     \\
            \texttt{ rorroeche@gmail.com }                                              \\[2.5ex]
			Longo Nicolás, \textit{Padrón Nro. 98271}                    
\\
            \texttt{ longo.gnr@hotmail.com }                                              \\[2.5ex]
            \normalsize{1er. Cuatrimestre de 2019}                                      \\
            \normalsize{66.20 Organización de Computadoras  $-$ Práctica Martes}  \\
            \normalsize{Facultad de Ingeniería, Universidad de Buenos Aires}            \\
       }
\date{7/5/19}


\begin{document}

\maketitle
\thispagestyle{empty}   % quita el número en la primer página
\newpage

\section{Enunciado}

\begin{enumerate}
\item Objetivos \\
Familiarizarse con el conjunto de instrucciones MIPS y el concepto de ABI,
extendiendo un programa que resuelva el problema descripto en la secci\'on 4.

\item Alcance \\
Este trabajo p\'actico es de elaboraci\'on grupal, evaluaci\'on individual, y de
car\'acter obligatorio para todos alumnos del curso.

\item Requisitos \\
El informe deber\'a ser entregado personalmente, por escrito, en la fecha
estipulada, con una car\'atula que contenga los datos completos de todos los
integrantes.
Adem\'as, es necesario que el trabajo pr\'actico incluya (entre otras cosas,
ver secci\'on 6), la presentaci\'on de los resultados obtenidos, explicando, cuando
corresponda, con fundamentos reales, las causas o razones de cada caso.

\item Descripci\'on \\
En este trabajo, vamos a implementar dos programas en assembly MIPS,
que nos permitan convertir informaci\'on almacenada en streams de texto, de
acuerdo a lo descripto en el trabajo anterior [1]: unix2dos, para transformar
archivos de texto de UNIX a DOS; y dos2unix para realizar la operaci\'on inversa.
Es decir, siguiendo la l\'inea de trabajo de los pr\'acticos anteriores, buscamos
generar dos binarios diferentes en vez de un \'unico programa que resuelva
ambos problemas. Adem\'as, por tratarse de un TP orientado a practicar el
conjunto de instrucciones, ambos programas deber\'an escritos completamente
en assembly, en forma manual.

\item Interfaz \\
Por defecto, ambos programas operan exclusivamente por stdin y stdout.
Al igual que en el trabajo anterior, al finalizar, estos programas retornar\'an
un valor nulo en caso de no detectar ning\'un problema; y, en caso contrario,
finalizar\'an con c\'odigo no nulo (por ejemplo 1).


\item Ejemplos \\
Cuando el documento de entrada es vac\'io, la salida deber\'ia serlo tambi\'en:
\begin{verbatim}
$ unix2dos </dev/null | wc -c
0
$ dos2unix </dev/null | wc -c
0
\end{verbatim}
Tambi\'en se aplican todos los casos de prueba definidos para el TP anterior.
Por ejemplo:
\begin{verbatim}
$ od -t c /tmp/dos.txt
0000000 U n o \r \n D o s \r \n T r e s \r \n
0000020
$ dos2unix -i /tmp/dos.txt -o - | od -t c
0000000 U n o \n D o s \n T r e s \n
0000015
\end{verbatim}
Por \'ultimo, podemos usar el programa que sigue para generar secuencias de
datos aleatorias, pasarlas a hexa, reconvertirlas al dominio original, y verificar
que la operaci\'on no haya alterado la informaci\'on:
\begin{verbatim}
$ n=0; while :; do
> n="‘expr $n + 1‘";
> head -c 100 </dev/urandom >/tmp/test.$n.u;
> unix2dos </tmp/test.$n.u >/tmp/test.$n.d || break;
> dos2unix </tmp/test.$n.d >/tmp/test.$n.2.u || break;
> diff -q /tmp/test.$n.u /tmp/test.$n.2.u || break;
> rm -f /tmp/test.$n.*;
> echo Ok: $n;
> done; echo Error: $n
Ok: 1
Ok: 2
Ok: 3
...
\end{verbatim}

\item Implementaci\'on \\
El programa a implementar deber\'a satisfacer algunos requerimientos m\'inimos,
que detallamos a continuaci\'on:


\begin{enumerate}
\item ABI \\
Ser\'a necesario que el c\'odigo presentado utilice la ABI explicada en la clase:
los argumentos correspondientes a los registros $a0-$a3 ser\'an almacenados
por el callee, siempre, en los 16 bytes dedicados de la secci\'on “function call
argument area” [2].

\item Casos de prueba\\
Es necesario que la implementaci\'on propuesta pase todos los casos incluidos
tanto en este enunciado como en el conjunto pruebas suministrado en el
informe del trabajo, los cuales deber\'an estar debidamente documentados y
justificados.

\item Documentaci\'on \\
El informe deber\'a incluir una descripci\'on detallada de las t\'ecnicas y procesos
de medici\'on empleados, y de todos los pasos involucrados en el desarrollo
del mismo, ya que forman parte de los objetivos principales del trabajo.
\end{enumerate}

\item Informe\\
El informe deber\'a incluir:
\begin{itemize}
\item Este enunciado;

\item Documentaci\'on relevante al disenio, implementaci\'on y medici\'on de las
caracter\'isticas del programa;

\item Las corridas de prueba, (secci\'on 5.2) con los comentarios pertinentes;

\item El c\'odigo fuente completo, en dos formatos: digitalizado1 e impreso en
papel. 
\end{itemize}

\item Fechas \\
Fecha de vencimiento: 7/5.

\item Referencias
Enunciado del primer trabajo pr\'actico, primer cuatrimestre de 2019.
http://groups.yahoo.com/groups/orga-comp/.

\end{enumerate}
\newpage

\section{Implementaci\'on}
Se implementaron dos funciones para poder convertir un archivo de DOS a UNIX y viceversa. La implementaci\'on se basó en un ciclo while que recorre el archivo byte por byte mientras consulta si está parado sobre un '$\backslash$r' o '$\backslash$n'. En caso de ser alguno de estos caracteres, se efectua la modificación correspondiente y sino, simplemente se escribe el mismo caracter en la salida.
Para lograr esto, decidimos dividir los programas en pequeñas funciones.Esto nos permiti\'o por un lado reducir la complejidad e ir teniendo seguridad de que las cosas que desarroll\'abamos funcionaban de la manera correcta. Y por otro lado, reutilizar funciones que aplicaban tanto a unix2dos como a dos2unix. Las funciones desarroladas fueron:

\begin{itemize}
\item getch: devuelve el caracter le\'ido por stdin y en caso de ser EOF devuelve -1.

\item putch: escribe el caracter que recibe por stdout.

\item isLF: devuelve 1 si el caracter es LF y 0 si no lo es.

\item isCR: devuelve 1 si el caracter es CR y 0 si no lo es.
\end{itemize}

\section{Utilización}

\section{Generaci\'on de binarios}

\section{Pruebas}

\section{C\'odigo fuente}
\subsection{unix2dos}

\subsection{dos2unix}

\section{Conclusiones}


\end{document}
